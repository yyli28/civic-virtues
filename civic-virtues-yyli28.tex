\documentclass[
  man,
  floatsintext,
  longtable,
  nolmodern,
  notxfonts,
  notimes,
  colorlinks=true,linkcolor=blue,citecolor=blue,urlcolor=blue]{apa7}

\usepackage{amsmath}
\usepackage{amssymb}



\usepackage[bidi=default]{babel}
\babelprovide[main,import]{english}


% get rid of language-specific shorthands (see #6817):
\let\LanguageShortHands\languageshorthands
\def\languageshorthands#1{}

\RequirePackage{longtable}
\RequirePackage{threeparttablex}

\makeatletter
\renewcommand{\paragraph}{\@startsection{paragraph}{4}{\parindent}%
	{0\baselineskip \@plus 0.2ex \@minus 0.2ex}%
	{-.5em}%
	{\normalfont\normalsize\bfseries\typesectitle}}

\renewcommand{\subparagraph}[1]{\@startsection{subparagraph}{5}{0.5em}%
	{0\baselineskip \@plus 0.2ex \@minus 0.2ex}%
	{-\z@\relax}%
	{\normalfont\normalsize\bfseries\itshape\hspace{\parindent}{#1}\textit{\addperi}}{\relax}}
\makeatother




\usepackage{longtable, booktabs, multirow, multicol, colortbl, hhline, caption, array, float, xpatch}
\setcounter{topnumber}{2}
\setcounter{bottomnumber}{2}
\setcounter{totalnumber}{4}
\renewcommand{\topfraction}{0.85}
\renewcommand{\bottomfraction}{0.85}
\renewcommand{\textfraction}{0.15}
\renewcommand{\floatpagefraction}{0.7}

\usepackage{tcolorbox}
\tcbuselibrary{listings,theorems, breakable, skins}
\usepackage{fontawesome5}

\definecolor{quarto-callout-color}{HTML}{909090}
\definecolor{quarto-callout-note-color}{HTML}{0758E5}
\definecolor{quarto-callout-important-color}{HTML}{CC1914}
\definecolor{quarto-callout-warning-color}{HTML}{EB9113}
\definecolor{quarto-callout-tip-color}{HTML}{00A047}
\definecolor{quarto-callout-caution-color}{HTML}{FC5300}
\definecolor{quarto-callout-color-frame}{HTML}{ACACAC}
\definecolor{quarto-callout-note-color-frame}{HTML}{4582EC}
\definecolor{quarto-callout-important-color-frame}{HTML}{D9534F}
\definecolor{quarto-callout-warning-color-frame}{HTML}{F0AD4E}
\definecolor{quarto-callout-tip-color-frame}{HTML}{02B875}
\definecolor{quarto-callout-caution-color-frame}{HTML}{FD7E14}

%\newlength\Oldarrayrulewidth
%\newlength\Oldtabcolsep


\usepackage{hyperref}




\providecommand{\tightlist}{%
  \setlength{\itemsep}{0pt}\setlength{\parskip}{0pt}}
\usepackage{longtable,booktabs,array}
\usepackage{calc} % for calculating minipage widths
% Correct order of tables after \paragraph or \subparagraph
\usepackage{etoolbox}
\makeatletter
\patchcmd\longtable{\par}{\if@noskipsec\mbox{}\fi\par}{}{}
\makeatother
% Allow footnotes in longtable head/foot
\IfFileExists{footnotehyper.sty}{\usepackage{footnotehyper}}{\usepackage{footnote}}
\makesavenoteenv{longtable}

\usepackage{graphicx}
\makeatletter
\def\maxwidth{\ifdim\Gin@nat@width>\linewidth\linewidth\else\Gin@nat@width\fi}
\def\maxheight{\ifdim\Gin@nat@height>\textheight\textheight\else\Gin@nat@height\fi}
\makeatother
% Scale images if necessary, so that they will not overflow the page
% margins by default, and it is still possible to overwrite the defaults
% using explicit options in \includegraphics[width, height, ...]{}
\setkeys{Gin}{width=\maxwidth,height=\maxheight,keepaspectratio}
% Set default figure placement to htbp
\makeatletter
\def\fps@figure{htbp}
\makeatother







\usepackage{newtx}

\defaultfontfeatures{Scale=MatchLowercase}
\defaultfontfeatures[\rmfamily]{Ligatures=TeX,Scale=1}





\title{The relationship between international study and civic virtues}


\shorttitle{The relationship between international study and civic
virtues}


\usepackage{etoolbox}






\author{Yangyue Li}



\affiliation{
{MA Program in the Social Sciences, University of Chicago}}




\leftheader{Li}



\abstract{Not ready}

\keywords{civic virtues, empathy, wisdom, civility}

\authornote{ 

\par{       }
\par{Correspondence concerning this article should be addressed
to Yangyue Li, MA Program in the Social Sciences, University of
Chicago, 1155 E 60th
St., Chicago, IL 60637, USA, Email: yangyueli28@uchicago.edu}
}

\makeatletter
\let\endoldlt\endlongtable
\def\endlongtable{
\hline
\endoldlt
}
\makeatother

\urlstyle{same}



\makeatletter
\@ifpackageloaded{caption}{}{\usepackage{caption}}
\AtBeginDocument{%
\ifdefined\contentsname
  \renewcommand*\contentsname{Table of contents}
\else
  \newcommand\contentsname{Table of contents}
\fi
\ifdefined\listfigurename
  \renewcommand*\listfigurename{List of Figures}
\else
  \newcommand\listfigurename{List of Figures}
\fi
\ifdefined\listtablename
  \renewcommand*\listtablename{List of Tables}
\else
  \newcommand\listtablename{List of Tables}
\fi
\ifdefined\figurename
  \renewcommand*\figurename{Figure}
\else
  \newcommand\figurename{Figure}
\fi
\ifdefined\tablename
  \renewcommand*\tablename{Table}
\else
  \newcommand\tablename{Table}
\fi
}
\@ifpackageloaded{float}{}{\usepackage{float}}
\floatstyle{ruled}
\@ifundefined{c@chapter}{\newfloat{codelisting}{h}{lop}}{\newfloat{codelisting}{h}{lop}[chapter]}
\floatname{codelisting}{Listing}
\newcommand*\listoflistings{\listof{codelisting}{List of Listings}}
\makeatother
\makeatletter
\makeatother
\makeatletter
\@ifpackageloaded{caption}{}{\usepackage{caption}}
\@ifpackageloaded{subcaption}{}{\usepackage{subcaption}}
\makeatother

% From https://tex.stackexchange.com/a/645996/211326
%%% apa7 doesn't want to add appendix section titles in the toc
%%% let's make it do it
\makeatletter
\xpatchcmd{\appendix}
  {\par}
  {\addcontentsline{toc}{section}{\@currentlabelname}\par}
  {}{}
\makeatother

%% Disable longtable counter
%% https://tex.stackexchange.com/a/248395/211326

\usepackage{etoolbox}

\makeatletter
\patchcmd{\LT@caption}
  {\bgroup}
  {\bgroup\global\LTpatch@captiontrue}
  {}{}
\patchcmd{\longtable}
  {\par}
  {\par\global\LTpatch@captionfalse}
  {}{}
\apptocmd{\endlongtable}
  {\ifLTpatch@caption\else\addtocounter{table}{-1}\fi}
  {}{}
\newif\ifLTpatch@caption
\makeatother

\begin{document}

\maketitle


\setcounter{secnumdepth}{-\maxdimen} % remove section numbering

\setlength\LTleft{0pt}


In an interconnected world, civic virtues are crucial for fostering
responsible citizenship and encouraging individuals to prioritize
societal well-being over self-interest (Sherrod et al., 2002). U.S.
universities, as diverse microcosms, bring domestic and international
students together, offering unique contexts for civic engagement and
understanding the values of others. Domestic American students may
develop civic engagement through strong ties to their local communities
and the American education system, while international students may
bring perspectives shaped by cross-cultural experiences and adaptation
to a new cultural environment. Undergoing the same admission process and
sharing similar college life on campus with their American peers,
international students may still differ in civic behaviors. These
contexts raise critical questions about the factors that foster civic
virtues and motivate individuals to engage in civic-minded behaviors.
How do experiences in a foreign and culturally different environment
influence civic virtues and related psychological characteristics such
as empathy and cultural competence? Do international students differ on
these psychological aspects from US students?

\subsection{Literature Review}\label{literature-review}

\subsubsection{Education and Civic
Virtues}\label{education-and-civic-virtues}

Civic engagement may foster a sense of belonging, purpose, and
responsibility within a community, and may empower individuals to
contribute to a common good (Flanagan \& Levine, 2010). Research
suggests that civic education - educational experiences that are
directed at increasing civic virtues - happens mostly during
adolescence, but can continue into young adulthood (Flanagan \& Levine,
2010; Sherrod et al., 2002). Much of this experience occurs in
educational settings, particularly colleges, which offer structured
opportunities through coursework, extracurricular activities, community
service programs, and student organizations (Flanagan \& Levine, 2010).
These experiences could help young adults develop civic virtues and
social responsibility.

Besides formal instruction in character education and civility (Jeynes,
2019; Torney-Purta, 2002), non-didactic and informal experiences have
important implications for the development of civic virtues as well.
Indeed, research highlights that youths who are involved in
community-based organizations or extracurricular activities are more
likely to be civically active (Zarrett et al., 2021). Furthermore, a
study suggests that participating in community service or volunteering
is associated with adolescent civic beliefs about considering similar
types of civic engagement behaviors (Metzger et al., 2019). Similarly,
Vazina \& Poulin (2019) found that prosocial/community-based activities
are related to a greater likelihood of being in the high-sustained civic
engagement trajectory.

Additionally, recent research suggests that study abroad programs may
increase some aspects of civic virtues and related psychological
capacities. Living and studying in a foreign cultural environment
provide opportunities for students to interact with people from diverse
backgrounds, fostering skills related to civic virtues such as empathy,
open-mindedness, and perspective-taking in a non-didactic way (Chieffo
\& Griffiths, 2004). Research indicates that even short-term study
abroad programs enhance student's empathy and open-mindedness toward
diverse perspectives, and foster a deeper sense of responsibility as
global citizens (Chieffo \& Griffiths, 2004). Black and Duhon (2006)
assessed the impact of a month-long business-focused study abroad
program in London on students' cultural awareness and personal
development. Consistent with previous findings, students showed
significant improvements in cross-cultural empathy and understanding of
global perspectives. A more recent study examined the relationship
between studying abroad and civic virtues (Boulware et al., 2023). The
results suggest that undergraduate students who have gone abroad via one
particular set of study abroad programs in undergraduate studies
demonstrated higher levels of empathy, civic engagement, and civility
toward others, compared with students who do not have these experiences
or have no interest in them.

\subsection{Current Study}\label{current-study}

The study examined how factors such as international students' first
language, English language fluency, and the duration of their residence
in the United States relate to civic virtues. We hypothesize that
international students with greater English fluency and who live in the
United States for a longer period of time will score higher on civic
virtues, as students with higher fluency may feel more confident and
able to engage in community activities and social settings. However, it
is unclear whether different native languages would interact with this.
Given that language differences are related to cultural differences, it
may be the case that English competence alone may not be significant. We
hypothesize that non-English-native speakers may score higher on civic
virtues than students who are native or heritage bilingual speakers of
English, maybe because navigating a foreign environment in a non-native
language may require greater effort and adaptation, fostering civic
virtues. Additionally, we also examined the possible interaction between
first language and English fluency to investigate whether non-native
English speakers who are more fluent in English score the highest civic
virtues.

\subsection{Method}\label{method}

\subsubsection{Participants}\label{participants}

A survey was adapted from Boulware et al.~(2023). Participants were
recruited through the Psychology Department's SONA system during the
winter quarter in 2025 as well as through campus advertisements.

\subsection{Analysis}\label{analysis}

\section{Data Cleaning}\label{data-cleaning}






\end{document}
